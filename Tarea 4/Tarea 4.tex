\documentclass{beamer}

% Tema y colores - usando un tema más compacto
\usetheme{Frankfurt}
\usecolortheme{whale}
\useoutertheme{miniframes}

% Paquetes
\usepackage[spanish]{babel}
\usepackage{amsmath}
\usepackage{graphicx}
\usepackage{tikz}
\usetikzlibrary{shapes,arrows,positioning,decorations.pathreplacing,calligraphy}

% Configuración para reducir tamaño de fuente y espacios
\setbeamerfont{title}{size=\small}
\setbeamerfont{subtitle}{size=\footnotesize}
\setbeamerfont{frametitle}{size=\normalsize}
\setbeamerfont{framesubtitle}{size=\footnotesize}
\setbeamerfont{itemize/enumerate body}{size=\footnotesize}
\setbeamerfont{itemize/enumerate subbody}{size=\scriptsize}
\setbeamerfont{caption}{size=\scriptsize}
\setbeamerfont{description item}{size=\footnotesize}
\setbeamerfont{normal text}{size=\footnotesize}

% Reducir márgenes y espaciado
\setbeamersize{text margin left=5pt,text margin right=5pt}
\setlength{\leftmargini}{10pt}
\setlength{\leftmarginii}{8pt}

% Reducir espaciado vertical
\setlength{\parskip}{1pt}
\renewcommand{\baselinestretch}{0.9}

% Información del título
\title[Inventario y Enrutamiento]{Resolución del Problema Integrado de Enrutamiento y Gestión de Inventarios}
\subtitle{Aplicación de Programación Lineal Entera Mixta}
\author[Estudiante]{Etzel Yuliza Peralta Lopez}
\institute[Universidad]{Universidad Nacional del Altiplano Puno}
% Eliminamos la fecha
\date{}

\begin{document}
	
	% Diapositiva del título
	\begin{frame}
		\titlepage
	\end{frame}
	
	% Diapositiva 1: Definición del Problema
	\begin{frame}{Definición del Problema}
		\begin{columns}
			\column{0.55\textwidth}
			\begin{itemize}
				\item \textbf{Problema Multi-Vehicle IRP}
				\item \textbf{Objetivo:} Minimizar costos de:
				\begin{itemize}
					\item Enrutamiento
					\item Gestión de inventario
				\end{itemize}
				\item \textbf{Consideraciones:}
				\begin{itemize}
					\item Múltiples vehículos
					\item Horizonte de planificación definido
					\item Demandas conocidas por cliente
					\item Capacidades limitadas (almacén/vehículos)
				\end{itemize}
			\end{itemize}
			
			\column{0.45\textwidth}
			\begin{figure}
				\centering
				\begin{tikzpicture}[scale=0.8]
					% Depósito central
					\node[rectangle,draw,fill=blue!20] (depot) at (0,0) {Depósito};
					
					% Clientes
					\node[circle,draw,fill=green!20] (c1) at (-2,2) {C1};
					\node[circle,draw,fill=green!20] (c2) at (2,2) {C2};
					\node[circle,draw,fill=green!20] (c3) at (-2,-2) {C3};
					\node[circle,draw,fill=green!20] (c4) at (2,-2) {C4};
					
					% Rutas - vehículo 1
					\draw[->,thick,red] (depot) -- (c1);
					\draw[->,thick,red] (c1) -- (c2);
					\draw[->,thick,red] (c2) -- (depot);
					
					% Rutas - vehículo 2
					\draw[->,thick,blue] (depot) -- (c3);
					\draw[->,thick,blue] (c3) -- (c4);
					\draw[->,thick,blue] (c4) -- (depot);
					
					% Leyenda
					\node[red] at (1.5,3) {Vehículo 1};
					\node[blue] at (1.5,-3) {Vehículo 2};
				\end{tikzpicture}
				\caption{Ejemplo de rutas con múltiples vehículos}
			\end{figure}
		\end{columns}
	\end{frame}
	
	% Diapositiva 2: Formulación Matemática
	\begin{frame}{Formulación Matemática}
		\begin{columns}
			\column{0.62\textwidth}
			\textbf{Función objetivo:}
			\begin{equation*}
				\min \sum_{i \in V}\sum_{t \in M}H_i I_{it} + \sum_{i,j \in V}\sum_{k \in K}\sum_{t \in T}Costo_{ij}X_{ijkt}
			\end{equation*}
			
			\vspace{0.2cm}
			\textbf{Variables de decisión:} {\small
				\begin{itemize}
					\item $X_{ijkt}$: Si el arco $ij$ es usado por vehículo $k$ en periodo $t$
					\item $I_{it}$: Nivel de inventario del cliente $i$ al final del periodo $t$
					\item $q_{ikt}$: Cantidad enviada al cliente $i$ por vehículo $k$ en $t$
					\item $Y_{ikt}$: Si el cliente $i$ es visitado por vehículo $k$ en $t$
			\end{itemize}}
			
			\vspace{0.2cm}
			\textbf{Restricciones principales:} {\small
				\begin{itemize}
					\item Balance de inventario: $I_{it} = I_{it-1} + \sum_{k \in K} q_{ikt} - d_{it}$
					\item Capacidad de almacenamiento: $I_{it} \leq C_i$
					\item Capacidad de vehículos: $\sum_{i \in V1} q_{ikt} \leq QY_{0kt}$
					\item Conservación de flujo: $\sum_{j \in V} X_{ijkt} = Y_{ikt}$
			\end{itemize}}
			
			\column{0.38\textwidth}
			\begin{figure}
				\centering
				\begin{tikzpicture}[scale=0.7]
					% Componentes del costo - rectangulos más largos
					\draw[fill=blue!20] (0,0) rectangle (5,0.8) node[pos=.5,font=\footnotesize] {Costo Total};
					
					\draw[fill=green!20] (0,1.8) rectangle (2.2,2.6) node[pos=.5,font=\footnotesize] {Inventario};
					\draw[fill=red!20] (2.8,1.8) rectangle (5,2.6) node[pos=.5,font=\footnotesize] {Transporte};
					
					\draw[->,thick] (1.1,1.8) -- (2.5,0.8);
					\draw[->,thick] (3.9,1.8) -- (2.5,0.8);
					
					% Variables
					\node[draw,ellipse,fill=yellow!20,font=\footnotesize] at (1.1,3.6) {$I_{it}$};
					\draw[->,thick] (1.1,3.2) -- (1.1,2.6);
					
					\node[draw,ellipse,fill=yellow!20,font=\footnotesize] at (3.9,3.6) {$X_{ijkt}$};
					\draw[->,thick] (3.9,3.2) -- (3.9,2.6);
				\end{tikzpicture}
				\caption{\footnotesize Estructura del modelo}
			\end{figure}
			\end{columns}
		\end{frame}
	
	% Diapositiva 3: Políticas de Gestión de Inventario
	\begin{frame}{Políticas de Gestión de Inventario}
		\begin{columns}[T]
			\column{0.48\textwidth}
			\textbf{Maximum Level (ML):} {\footnotesize
				\begin{itemize}\setlength{\itemsep}{1pt}
					\item Repone según demanda del cliente
					\item No excede capacidad máxima
					\item \textbf{Ecuaciones clave:}
				\end{itemize}
				\begin{align*}
					\sum_{k \in K} q_{ikt} &\leq C_i - I_{it-1}\\
					q_{ikt} &\leq C_i Y_{ikt}
			\end{align*}}
			
			\column{0.48\textwidth}
			\textbf{Order Up to Level (OU):} {\footnotesize
				\begin{itemize}\setlength{\itemsep}{1pt}
					\item Llena completamente el inventario
					\item Alcanza nivel máximo de almacén
					\item \textbf{Ecuación adicional:}
				\end{itemize}
				\begin{align*}
					q_{ikt} \geq C_i Y_{it} - I_{it-1}
			\end{align*}}
		\end{columns}
		
		\vspace{0.2cm}
		\begin{figure}
			\centering
			\begin{tikzpicture}[scale=0.65]
				% ML Policy - más a la izquierda
				\begin{scope}[xshift=-0.5cm]
					\draw[thick] (0,0) -- (4,0);
					\draw[thick] (0,0) -- (0,2);
					\node[below,font=\footnotesize] at (2,0) {Tiempo};
					\node[left, rotate=90,font=\footnotesize] at (-0.5,1) {Inventario};
					
					\draw[thick,blue!50,fill=blue!20] (0.2,0) rectangle (0.8,0.8);
					\draw[thick,blue!50,fill=blue!20] (1.2,0) rectangle (1.8,1.2);
					\draw[thick,blue!50,fill=blue!20] (2.2,0) rectangle (2.8,0.6);
					\draw[thick,blue!50,fill=blue!20] (3.2,0) rectangle (3.8,1.0);
					
					\draw[dashed] (0,1.5) -- (4,1.5);
					\node[right,font=\footnotesize] at (4,1.5) {Máx};
					\node[below,font=\footnotesize] at (2,-0.5) {ML: Variable};
				\end{scope}
				
				% OU Policy - más espacio entre gráficos
				\begin{scope}[xshift=6.5cm]
					\draw[thick] (0,0) -- (4,0);
					\draw[thick] (0,0) -- (0,2);
					\node[below,font=\footnotesize] at (2,0) {Tiempo};
					
					\draw[thick,red!50,fill=red!20] (0.2,0) rectangle (0.8,1.5);
					\draw[thick,red!50,fill=red!20] (1.2,0) rectangle (1.8,1.5);
					\draw[thick,red!50,fill=red!20] (2.2,0) rectangle (2.8,1.5);
					\draw[thick,red!50,fill=red!20] (3.2,0) rectangle (3.8,1.5);
					
					\draw[dashed] (0,1.5) -- (4,1.5);
					\node[right,font=\footnotesize] at (4,1.5) {Máx};
					\node[below,font=\footnotesize] at (2,-0.5) {OU: Siempre máximo};
				\end{scope}
			\end{tikzpicture}
			\caption{\footnotesize Comparación de políticas de inventario}
		\end{figure}
	\end{frame}
	
	% Diapositiva 4: Eliminación de Sub-tours
	\begin{frame}{Eliminación de Sub-tours}
		\begin{columns}[T]
			\column{0.48\textwidth}
			\textbf{Tres métodos para eliminar sub-tours:} {\footnotesize
				\begin{enumerate}\setlength{\itemsep}{1pt}
					\item \textbf{Modelo General:} Variables de carga acumulada
					\begin{align*}
						U_{jkt} \geq U_{ikt} + q_{jkt} + QX_{ijkt} - Q
					\end{align*}
					
					\item \textbf{Modelo de Flujos:} Variables de flujo $f_{ijkt}$
					\begin{align*}
						\sum_{j} f_{0jkt} &= \sum_{j} Y_{jkt} \\
						\sum_{j} f_{ijkt} &= \sum_{j} f_{jikt} - Y_{ikt}
					\end{align*}
					
					\item \textbf{Método MTZ:} Define orden de visita
					\begin{align*}
						W_{0kt} &= 0 \\
						W_{jkt} &\geq W_{ikt} + 1 - n(1 - X_{ijkt})
					\end{align*}
			\end{enumerate}}
			
			\column{0.48\textwidth}
			\begin{minipage}{\textwidth}
				\centering
				\textbf{Problema de sub-tours}
				
				\begin{tikzpicture}[scale=0.55]
					% Tour principal (D-A-B-D)
					\draw[thick] (0,1) circle (0.25);
					\node[font=\footnotesize] at (0,1) {D};
					
					\draw[thick] (1.5,2) circle (0.25);
					\node[font=\footnotesize] at (1.5,2) {A};
					
					\draw[thick] (3,1) circle (0.25);
					\node[font=\footnotesize] at (3,1) {B};
					
					\draw[->,thick] (0.25,1.1) -- (1.3,1.9);
					\draw[->,thick] (1.7,1.9) -- (2.8,1.2);
					\draw[->,thick] (2.9,0.8) -- (0.1,0.8);
					
					% Sub-tour separado (C-E-C)
					\draw[thick] (0.5,-0.5) circle (0.25);
					\node[font=\footnotesize] at (0.5,-0.5) {C};
					
					\draw[thick] (2.5,-0.5) circle (0.25);
					\node[font=\footnotesize] at (2.5,-0.5) {E};
					
					\draw[->,thick,red] (0.75,-0.5) -- (2.25,-0.5);
					\draw[->,thick,red] (2.5,-0.75) -- (2.5,-1.25) -- (0.5,-1.25) -- (0.5,-0.75);
					
					% Etiquetas de tours
					\node[above,font=\scriptsize] at (1.5,2.2) {Tour 1};
					\node[below,font=\scriptsize] at (1.5,-1.4) {Tour 2 (sub-tour)};
				\end{tikzpicture}
				
				\vspace{0.15cm}
				\textbf{Solución correcta (ruta única)}
				
				\begin{tikzpicture}[scale=0.55]
					% Una sola ruta conectada
					\draw[thick] (1.5,1.5) circle (0.25);
					\node[font=\footnotesize] at (1.5,1.5) {D};
					
					\draw[thick] (3,2.5) circle (0.25);
					\node[font=\footnotesize] at (3,2.5) {A};
					
					\draw[thick] (4.5,1.5) circle (0.25);
					\node[font=\footnotesize] at (4.5,1.5) {B};
					
					\draw[thick] (0,0) circle (0.25);
					\node[font=\footnotesize] at (0,0) {C};
					
					\draw[thick] (3,0) circle (0.25);
					\node[font=\footnotesize] at (3,0) {E};
					
					% Ruta conectada
					\draw[->,thick,blue] (1.75,1.6) -- (2.8,2.4);
					\draw[->,thick,blue] (3.2,2.4) -- (4.3,1.7);
					\draw[->,thick,blue] (4.4,1.25) -- (3.2,0.2);
					\draw[->,thick,blue] (2.75,0) -- (0.25,0);
					\draw[->,thick,blue] (0,0.25) -- (1.35,1.35);
					
					% Etiqueta de ruta única
					\node[above,font=\scriptsize] at (2.25,2.8) {Ruta única (D-A-B-E-C-D)};
				\end{tikzpicture}
			\end{minipage}
		\end{columns}
	\end{frame}
	
	% Diapositiva 5: Resultados y Conclusiones
	\begin{frame}{Resultados y Conclusiones}
		\begin{columns}[T]
			\column{0.55\textwidth}
			\textbf{Resultados principales:}
			
			\begin{minipage}{\textwidth}
				{\footnotesize
					\begin{center}
						\begin{tabular}{|l|c|c|c|}
							\hline
							\textbf{Método} & \textbf{Flujos} & \textbf{MTZ} & \textbf{General} \\
							\hline
							GAP promedio & 1.49\% & 6.62\% & 9.21\% \\
							Tiempo (s) & 655 & 728 & 1123 \\
							\hline
						\end{tabular}
				\end{center}}
			\end{minipage}
			
			\vspace{0.2cm}
			{\footnotesize \textbf{Hallazgos:}}
			\begin{itemize}\setlength{\itemsep}{0pt}
				\item \textbf{Mejor:} Modelo de Flujos + política ML
				\item \textbf{Intermedio:} Método MTZ 
				\item \textbf{Menor rendimiento:} Modelo General
			\end{itemize}
			
			\vspace{0.2cm}
			{\footnotesize \textbf{Conclusiones:}}
			\begin{itemize}\setlength{\itemsep}{0pt}
				\item Metodología adaptable a diferentes aplicaciones
				\item Buena calidad en instancias pequeñas/medianas
				\item Limitante: Alta complejidad computacional
			\end{itemize}
			
			\vspace{0.2cm}
			{\footnotesize \textbf{Trabajo futuro:}}
			\begin{itemize}\setlength{\itemsep}{0pt}
				\item Desarrollo de heurísticas y metaheurísticas
				\item Modelos con demandas estocásticas
				\item Manejo de productos perecederos
			\end{itemize}
			
			\column{0.45\textwidth}
			\begin{tikzpicture}[scale=0.6]
				% Título
				\node[font=\footnotesize] at (2.5,4.5) {\textbf{Comparativa de métodos}};
				
				% Gráfico de barras - Tiempo computacional
				\begin{scope}
					\draw[thick] (0,0) -- (5,0);
					\draw[thick] (0,0) -- (0,3);
					\node[above,font=\footnotesize] at (2.5,3.2) {Tiempo (s)};
					
					\draw[fill=red!50] (0.8,0) rectangle (1.7,3);
					\draw[fill=blue!50] (2.2,0) rectangle (3.1,2);
					\draw[fill=green!50] (3.6,0) rectangle (4.5,1);
					
					% Etiquetas con valores
					\node[font=\scriptsize] at (1.25,3.3) {1123};
					\node[font=\scriptsize] at (2.65,2.3) {728};
					\node[font=\scriptsize] at (4.05,1.3) {655};
					
					% Etiquetas de métodos
					\node[below,font=\scriptsize] at (1.25,0) {General};
					\node[below,font=\scriptsize] at (2.65,0) {MTZ};
					\node[below,font=\scriptsize] at (4.05,0) {Flujos};
				\end{scope}
				
				% Gráfico de barras - GAP (con más separación)
				\begin{scope}[yshift=-5.5cm]
					\draw[thick] (0,0) -- (5,0);
					\draw[thick] (0,0) -- (0,3);
					\node[above,font=\footnotesize] at (2.5,3.2) {GAP (\%)};
					
					\draw[fill=red!50] (0.8,0) rectangle (1.7,2.8);
					\draw[fill=blue!50] (2.2,0) rectangle (3.1,2);
					\draw[fill=green!50] (3.6,0) rectangle (4.5,0.7);
					
					% Etiquetas con valores
					\node[font=\scriptsize] at (1.25,3.1) {9.21\%};
					\node[font=\scriptsize] at (2.65,2.3) {6.62\%};
					\node[font=\scriptsize] at (4.05,1) {1.49\%};
					
					% Etiquetas de métodos
					\node[below,font=\scriptsize] at (1.25,0) {General};
					\node[below,font=\scriptsize] at (2.65,0) {MTZ};
					\node[below,font=\scriptsize] at (4.05,0) {Flujos};
				\end{scope}
			\end{tikzpicture}
		\end{columns}
	\end{frame}
	
	\end{document}