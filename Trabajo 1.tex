\documentclass[]{article}

%opening
\usepackage[utf8]{inputenc}
\usepackage[spanish]{babel}
\usepackage{amsmath}
\usepackage{amsfonts}
\usepackage{amssymb}
\usepackage{enumitem}
\usepackage{hyperref} 

\title{\textbf{Variable: Productividad Laboral}}
\date{}

\begin{document}
	
	\maketitle
	\section*{Definición de la variable}
	
	La productividad laboral significa cuánto trabajo puede hacer una persona en un cierto período de tiempo. Se mide como la relación entre la producción total y las horas trabajadas. Esta variable es fundamental para evaluar la eficiencia de los empleados y el rendimiento general de una organización.
	
	\section*{Función Matemática}
	
	La productividad laboral se puede expresar mediante la siguiente función matemática:
	
	\[
	P = \frac{Q}{T}
	\]
	
	Donde:
	\begin{itemize}
		\item $P$ es la productividad
		\item $Q$ es la cantidad de unidades producidas
		\item $T$ es el tiempo de trabajo (puede ser en horas, días, etc.)
	\end{itemize}
	
	\section*{Función}
	
	La productividad laboral desempeña varias funciones en una organización, tales como:
	
	\begin{enumerate}
		\item \textbf{Evaluación de Eficiencia:} Permite a las empresas medir cuán eficientemente están utilizando sus recursos humanos.
		\item \textbf{Toma de Decisiones:} Ayuda a los gerentes a identificar áreas que necesitan mejoras y a tomar decisiones informadas sobre la asignación de recursos.
		\item \textbf{Comparación:} Facilita la comparación de la productividad entre diferentes departamentos o con otras empresas del mismo sector.
	\end{enumerate}
	
	\section*{Restricción}
	
	La productividad se medirá en función de la cantidad de unidades producidas por hora trabajada en una fábrica a lo largo de una semana.
	
	\section*{Ejemplo Práctico}
	
	Imaginemos que una empresa quiere medir la productividad de dos equipos de trabajo en una semana. Cada equipo tiene una cantidad diferente de trabajadores y produce diferentes cantidades de unidades.
	
	\begin{itemize}
		\item \textbf{Equipo A} tiene 10 empleados y en total produce 2500 unidades trabajando durante 500 horas. La productividad de este equipo se calcula de la siguiente manera:
		\[
		P = \frac{2500 \text{ unidades}}{500 \text{ horas}} = 5 \text{ unidades por hora}
		\]
		\item \textbf{Equipo B} tiene 8 empleados y produce 2000 unidades trabajando 400 horas. Su productividad sería:
		\[
		P = \frac{2000 \text{ unidades}}{400 \text{ horas}} = 5 \text{ unidades por hora}
		\]
	\end{itemize}
	
	Ambos equipos tienen una productividad de 5 unidades por hora. Sin embargo, el equipo B tiene menos trabajadores, lo que significa que usa mejor su personal para producir las mismas unidades en menos horas. Esto indica que, aunque ambos equipos son igualmente productivos, el equipo B es más eficiente al emplear menos recursos (trabajadores) para lograr la misma cantidad de producción.
	
	\section*{Referencias Bibliográficas}
	
	\begin{thebibliography}{}
		
		\bibitem{chiavenato2007}
		Chiavenato, I. (2007). \textit{Administración de recursos humanos} (6ª ed.). McGraw-Hill Interamericana.
		
		\bibitem{inei2023}
		Instituto Nacional de Estadística e Informática – INEI. (2023). \textit{Indicadores de productividad laboral en el Perú: Informe técnico}. Recuperado de \url{https://www.inei.gob.pe}
		
		\bibitem{robbins2018}
		Robbins, S. P., \& Coulter, M. (2018). \textit{Administración} (14ª ed.). Pearson Educación.
		
		\bibitem{gomez2016}
		Gómez-Mejía, L. R., Balkin, D. B., \& Cardy, R. L. (2016). \textit{Gestión de recursos humanos} (6ª ed.). Pearson.
		
	\end{thebibliography}
	
\end{document} 
