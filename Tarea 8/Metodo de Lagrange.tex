\documentclass{beamer}
\usepackage[utf8]{inputenc}
\usepackage[spanish]{babel}
\usepackage{amsmath, amssymb}
\usepackage{graphicx}
\usepackage{tikz}
\usepackage{multicol}
\usetheme{Madrid}
\usecolortheme{seahorse}

\title{\textbf{El Método de Lagrange} \\ \large Aplicando la Optimización}
\author{}
\date{}

\begin{document}
	
	% PORTADA
	\begin{frame}
		\titlepage
	\end{frame}
	
	% INTRODUCCIÓN
	\begin{frame}{¿Qué es el Método de Lagrange?}
		\begin{itemize}
			\item Técnica matemática para optimizar funciones con restricciones.
			\item Se usa cuando se desea maximizar o minimizar \( f(x, y, \dots) \) sujeto a \( g(x, y, \dots) = c \).
			\item Se introduce una nueva variable: el \textbf{multiplicador de Lagrange} \( \lambda \).
		\end{itemize}
	\end{frame}
	
	% PASOS DEL MÉTODO
	\begin{frame}{¿Cómo funciona el Método?}
		\begin{block}{Pasos generales}
			\begin{enumerate}
				\item Definir función objetivo \( f(x, y) \) y restricción \( g(x, y) = c \).
				\item Construir:
				\[
				\mathcal{L}(x, y, \lambda) = f(x, y) - \lambda (g(x, y) - c)
				\]
				\item Derivar: \( \frac{\partial \mathcal{L}}{\partial x} = 0, \frac{\partial \mathcal{L}}{\partial y} = 0, \frac{\partial \mathcal{L}}{\partial \lambda} = 0 \)
				\item Resolver el sistema de ecuaciones.
			\end{enumerate}
		\end{block}
	\end{frame}
	
	% EJEMPLO MATEMÁTICO
	\begin{frame}{Ejemplo Matemático Paso a Paso}
		\textbf{Maximizar:} \( f(x, y) = xy \quad \) \textbf{Sujeto a:} \( x + y = 10 \)
		
		\vspace{0.5em}
		\begin{block}{Lagrangiana}
			\[
			\mathcal{L}(x, y, \lambda) = xy - \lambda(x + y - 10)
			\]
		\end{block}
		
		\begin{block}{Derivadas}
			\begin{align*}
				\frac{\partial \mathcal{L}}{\partial x} &= y - \lambda = 0 \Rightarrow y = \lambda \\
				\frac{\partial \mathcal{L}}{\partial y} &= x - \lambda = 0 \Rightarrow x = \lambda \\
				\frac{\partial \mathcal{L}}{\partial \lambda} &= -(x + y - 10) = 0 \Rightarrow x + y = 10
			\end{align*}
		\end{block}
		
		\begin{block}{Resultado}
			\[
			x = y = 5 \Rightarrow f(5,5) = 25
			\]
		\end{block}
	\end{frame}
	
	% GRÁFICO
	\begin{frame}{Visualización Geométrica}
		\centering
		\begin{tikzpicture}[scale=1]
			\draw[->] (0,0) -- (5.5,0) node[right] {\(x\)};
			\draw[->] (0,0) -- (0,5.5) node[above] {\(y\)};
			\draw[blue, thick] (0,5) -- (5,0) node[below right] {\(x + y = 5\)};
			\draw[red, dashed, domain=0.8:4.5] plot (\x,{2.5/\x});
			\node[blue] at (2.3,2.5) {\scriptsize Restricción};
			\node[red] at (3.2,1.4) {\scriptsize Curvas \(f(x,y)\)};
		\end{tikzpicture}
		
		\vspace{0.5em}
		\textit{El óptimo se encuentra cuando la curva de nivel es tangente a la restricción.}
	\end{frame}
	
	% EJEMPLO REAL
	\begin{frame}{Ejemplo Real: Publicidad}
		\begin{columns}
			\column{0.5\textwidth}
			\begin{block}{Datos}
				\begin{itemize}
					\item Presupuesto: S/ 1000
					\item Redes sociales (\(x\)): 3 clientes/sol
					\item Radio (\(y\)): 2 clientes/sol
				\end{itemize}
			\end{block}
			
			\column{0.5\textwidth}
			\begin{block}{Modelo}
				\[
				x + y = 1000,\quad f(x, y) = 3x + 2y
				\]
			\end{block}
		\end{columns}
		
		\vspace{0.5em}
		\begin{block}{Resultado}
			\[
			\lambda_x = 3,\quad \lambda_y = 2 \Rightarrow 3 \ne 2
			\Rightarrow \text{ invertir todo en } x
			\]
			\[
			x = 1000,\quad y = 0 \Rightarrow f = 3000 \text{ clientes}
			\]
		\end{block}
	\end{frame}
	
	% INTERPRETACIÓN DE LAMBDA
	\begin{frame}{¿Qué representa el multiplicador \( \lambda \)?}
		\begin{itemize}
			\item Mide el cambio en el valor óptimo si se relaja la restricción.
			\item Ejemplo: si \( \lambda = 5 \), aumentar en 1 unidad el recurso genera 5 unidades más de beneficio.
			\item En economía: se interpreta como el \textbf{valor marginal}.
		\end{itemize}
	\end{frame}
	
	% APLICACIONES
	\begin{frame}{Aplicaciones del Método de Lagrange}
		\begin{multicols}{2}
			\begin{itemize}
				\item Economía
				\item Ingeniería
				\item Machine Learning
				\item Logística
				\item Transporte y distribución
				\item Diseño estructural
			\end{itemize}
		\end{multicols}
		\vspace{1em}
		\centering
		\textit{Utilizado donde hay que optimizar bajo condiciones reales.}
	\end{frame}
	
	% AMPLIACIONES
	\begin{frame}{Ampliaciones del Método}
		\begin{itemize}
			\item Si hay más restricciones:
			\[
			\mathcal{L} = f - \lambda_1(g_1 - c_1) - \lambda_2(g_2 - c_2)
			\]
			\item Base del método de Karush-Kuhn-Tucker (KKT)
			\item Ampliamente usado en optimización no lineal y programación convexa.
		\end{itemize}
	\end{frame}
	
	% CONCLUSIÓN
	\begin{frame}{Conclusión}
		\begin{itemize}
			\item El Método de Lagrange es clave para optimizar con restricciones.
			\item Se basa en la tangencia de gradientes: \( \nabla f = \lambda \nabla g \)
			\item Tiene aplicación práctica en múltiples disciplinas.
		\end{itemize}
		\vspace{1em}
		\centering
		\textbf{Una herramienta esencial para el análisis y la toma de decisiones.}
	\end{frame}
	
	% FINAL
	\begin{frame}
		\centering
		\Huge ¡Gracias por tu atención!
	\end{frame}
	
\end{document}
