\documentclass[12pt]{article}
\usepackage[utf8]{inputenc}
\usepackage[spanish]{babel}
\usepackage{amsmath}
\usepackage{amsfonts}
\usepackage{geometry}
\usepackage{hyperref}
\usepackage{graphicx}
\geometry{margin=2.5cm}

\title{Sistema de Ecuaciones Lineales}
\author{
	Angello Marcelo Zamora Valencia\thanks{Universidad Nacional del Altiplano Puno} \and 
	Mario Wilfredo Ramirez Puma\thanks{Facultad de Ingeniería Estadística e Informática} \and 
	Etzel Yuliza Peralta Lopez \and 
	Nestor Ademir Ruelas Yana
}
\date{} % Fecha eliminada

\begin{document}
	
	\maketitle
	
	\section{Introducción}
	
	Los sistemas de ecuaciones lineales son fundamentales en álgebra lineal y tienen aplicaciones en diversas áreas como ingeniería, física, economía y más. Un sistema de ecuaciones lineales puede representarse matricialmente como $Ax = b$, donde $A$ es la matriz de coeficientes, $x$ es el vector de incógnitas y $b$ es el vector de términos independientes.
	
	\section{Explicación del Programa}
	
	El programa implementado en Python utiliza la biblioteca \texttt{NumPy} para resolver sistemas de ecuaciones lineales. A continuación se detalla su funcionamiento:
	
	\subsection{Entrada de Datos}
	
	\begin{itemize}
		\item El usuario ingresa el número de incógnitas ($n$).
		\item Se solicitan los coeficientes de cada ecuación, que se almacenan en la matriz $A$.
		\item Se ingresan los términos independientes, que se almacenan en el vector $b$.
	\end{itemize}
	
	\subsection{Procesamiento}
	
	\begin{itemize}
		\item Los datos se convierten en \textit{arrays} de NumPy para su manipulación matricial.
		\item Se utiliza la función \texttt{np.linalg.solve()} para resolver el sistema.
	\end{itemize}
	
	\subsection{Salida de Resultados}
	
	\begin{itemize}
		\item Si el sistema tiene solución única, se muestran los valores de las incógnitas.
		\item Si el sistema no tiene solución única (matriz singular o incompatible), se muestra un mensaje de error.
	\end{itemize}
	
	\section{Teoría Matemática}
	
	Un sistema de ecuaciones lineales puede tener:
	
	\begin{itemize}
		\item \textbf{Solución única:} Cuando la matriz $A$ es invertible (determinante no nulo).
		\item \textbf{Infinitas soluciones:} Cuando el sistema es compatible indeterminado.
		\item \textbf{Ninguna solución:} Cuando el sistema es incompatible.
	\end{itemize}
	
	El método implementado utiliza descomposición LU internamente, que es numéricamente estable para la mayoría de matrices.
	
	\section{Casos de Uso}
	
	Este programa puede utilizarse para:
	
	\begin{itemize}
		\item Resolver problemas de circuitos eléctricos (leyes de Kirchhoff).
		\item Balancear ecuaciones químicas.
		\item Modelar problemas económicos (oferta y demanda).
		\item Resolver sistemas mecánicos (equilibrio de fuerzas).
	\end{itemize}
	
	\section{Ejemplo de Ejecución}
	
	Para el sistema:
	
	\[
	\begin{cases}
		2x + y = 5 \\
		x - y = 1
	\end{cases}
	\]
	
	El usuario ingresaría:
	
	\begin{itemize}
		\item Número de incógnitas: 2
		\item Coeficientes ecuación 1: 2 1
		\item Coeficientes ecuación 2: 1 -1
		\item Términos independientes: 5 1
	\end{itemize}
	
	La solución sería: $x = 2$, $y = 1$.
	
	\section{Bibliografía}
	
	\begin{enumerate}
		\item Strang, G. (2016). \textit{Introduction to Linear Algebra}. Wellesley-Cambridge Press.
		\item Burden, R. L., \& Faires, J. D. (2010). \textit{Numerical Analysis} (9th ed.). Cengage Learning.
		\item NumPy Documentation. \texttt{numpy.linalg.solve}. \url{https://numpy.org/doc/stable/reference/generated/numpy.linalg.solve.html}
		\item Anton, H., \& Rorres, C. (2010). \textit{Elementary Linear Algebra: Applications Version} (10th ed.). Wiley.
	\end{enumerate}
	
\end{document}
