\documentclass[12pt]{article}
\usepackage[utf8]{inputenc}
\usepackage[spanish]{babel}
\usepackage{graphicx}
\usepackage{amsmath}
\usepackage{float}
\usepackage{geometry}
\usepackage{xcolor}
\usepackage{titlesec}
\usepackage{booktabs}

\geometry{margin=2.5cm}

\titleformat{\section}
{\normalfont\Large\bfseries}{\thesection}{1em}{}

\title{Predicción de Montos Girados con Aprendizaje Automático}
\author{Etzel Yuliza Peralta López}
\date{}

\begin{document}
	
	\maketitle
	
	\section*{Resultados del Modelo de Predicción}
	
	El modelo fue entrenado utilizando técnicas de optimización de hiperparámetros con \textbf{Optuna}, aplicando diferentes algoritmos de regresión. El algoritmo que obtuvo el mejor desempeño fue \textbf{Ridge}.
	
	\vspace{1em}
	\noindent\textbf{Métricas de Evaluación}:
	\begin{itemize}
		\item Mejor algoritmo: \textbf{Ridge}
		\item Mejor score (error cuadrático medio negativo): \textbf{-240,510,864,901.58}
		\item Coeficiente de determinación \( R^2 \): \textbf{0.999}
		\item Error promedio absoluto: \textbf{0.19 millones de soles}
	\end{itemize}
	
	\vspace{1em}
	Estos resultados indican que el modelo tiene un desempeño \textbf{excelente} al predecir los montos girados en función del presupuesto institucional modificado (PIM), departamento, sector y año.
	
	\section*{Visualización del Desempeño del Modelo}
	
	\begin{figure}[H]
		\centering
		\includegraphics[width=0.95\textwidth]{img/Imagen1.png}
		\caption{Visualización del rendimiento del modelo: predicción vs. valores reales, análisis de errores y evolución del score durante la optimización.}
	\end{figure}
	
	\vspace{1em}
	
	\begin{figure}[H]
		\centering
		\includegraphics[width=0.7\textwidth]{img/Imagen2.png}
		\caption{Resultados finales: R², error promedio y validación del rendimiento del mejor modelo.}
	\end{figure}
	
	\vspace{1em}
	\noindent\textbf{Interpretación de los gráficos:}
	\begin{itemize}
		\item \textbf{Predicción vs Real:} Los datos se alinean sobre la línea ideal, lo cual indica un ajuste casi perfecto.
		\item \textbf{Distribución de errores:} Centrada cerca de cero y con baja dispersión.
		\item \textbf{Evolución del score:} El proceso de búsqueda de Optuna logra converger rápidamente.
		\item \textbf{Resumen final:} Se valida el rendimiento del modelo y se muestra su capacidad predictiva.
	\end{itemize}
	
	\section*{Interpretación Final}
	
	\textcolor{green!50!black}{\textbf{Excelente}}. El modelo presenta un altísimo nivel de precisión. Con un \( R^2 = 0.999 \) y un error promedio de apenas \textbf{0.19 millones de soles}, constituye una herramienta poderosa para apoyar la planificación y el análisis presupuestal.
	
	\vspace{1em}
	\textcolor{blue!70!black}{\textbf{El modelo está listo para ser utilizado}} con fines predictivos en la gestión de recursos públicos.
	
\end{document}
