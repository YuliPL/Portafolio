\documentclass{beamer}

% Codificación y lenguaje
\usepackage[utf8]{inputenc}
\usepackage[spanish]{babel}
\usepackage[T1]{fontenc}
\usepackage{lmodern}

% Tema y esquema de color
\usetheme{Madrid}              % Tema elegante y profesional
\usecolortheme{seagull}       % Esquema de color suave

% Personalización de elementos
\setbeamertemplate{navigation symbols}{}  % Quitar íconos de navegación
\setbeamercolor{title}{fg=white,bg=blue!70!black}
\setbeamercolor{frametitle}{fg=white,bg=blue!50!black}
\setbeamercolor{block title}{bg=blue!20!white, fg=black}
\setbeamercolor{block body}{bg=blue!5!white, fg=black}
\setbeamertemplate{itemize items}[circle] % Círculos para ítems

% Información del documento
\title[PRESENTACIONES EFECTIVAS]{\textbf{PRESENTACIONES ORALES EFECTIVAS}}
\author{Universidad de Comunicación}
\date{\today}

\begin{document}
	
	% Diapositiva 1: Portada
	\begin{frame}[plain]
		\titlepage
	\end{frame}
	
	% Diapositiva 2: Introducción
	\begin{frame}
		\frametitle{Introducción}
		
		\begin{block}{Características fundamentales}
			\begin{itemize}
				\item Presentaciones orales son intrínsecamente sincrónicas
				\item Imponen ritmo y secuencia (no hay retroceso como en textos)
				\item Comunicación no verbal enriquece el mensaje
				\item Experiencia presencial irreemplazable
				\item Un mensaje central claro es esencial
			\end{itemize}
		\end{block}
		
		\vspace{0.5em}
		
		\alert{\textbf{Clave del éxito:}} Las presentaciones orales efectivas requieren planificación estratégica enfocada en las necesidades de la audiencia.
	\end{frame}
	
	% Diapositiva 3: Estructura - Apertura
	\begin{frame}
		\frametitle{Estructura: La Apertura}
		
		\begin{block}{Componentes clave}
			\begin{itemize}
				\item Captador de atención pertinente
				\item Plantear el problema y objetivo
				\item Breve pero impactante (60–90 segundos)
				\item Conectar con los intereses de la audiencia
				\item Evitar fórmulas y saludos innecesarios
			\end{itemize}
		\end{block}
		
		\vspace{0.5em}
		\alert{\textbf{Error común:}} Muchos presentadores desperdician los primeros momentos cruciales con saludos genéricos que no aportan valor.
	\end{frame}
	
	% Diapositiva 4: Cuerpo y Cierre
	\begin{frame}
		\frametitle{Estructura: Cuerpo y Cierre}
		
		\begin{block}{Elementos del cuerpo}
			\begin{itemize}
				\item Evidencia clara y jerarquizada
				\item Vista previa de 2–5 puntos principales
				\item Transiciones lógicas entre secciones
				\item Señalización constante para orientación
			\end{itemize}
		\end{block}
		
		\begin{block}{Elementos del cierre}
			\begin{itemize}
				\item Recapitular puntos clave sintéticamente
				\item Final claro y convincente
				\item Repasar lo importante antes de concluir
				\item Conectar con la apertura (estructura circular)
			\end{itemize}
		\end{block}
	\end{frame}
	
	% Diapositiva 5: Recomendaciones Prácticas
	\begin{frame}
		\frametitle{Recomendaciones Prácticas}
		
		\begin{block}{Estrategias para el éxito}
			\begin{itemize}
				\item Brindar un "mapa" claro de la exposición
				\item Mantener señalización constante durante toda la presentación
				\item Reconocer aportes y colaboraciones clave
				\item Conectar con la audiencia mediante contacto visual
				\item Crear redundancia efectiva a través de vista previa y repaso
				\item Hablar sobre el tema, no sobre la presentación
				\item Practicar transiciones y momentos clave
			\end{itemize}
		\end{block}
		
		\vspace{0.5em}
		\alert{\textbf{Conclusión:}} Las presentaciones orales efectivas equilibran estructura, contenido y ejecución, enfocándose siempre en las necesidades de la audiencia.
	\end{frame}
	
\end{document}
